\documentclass[a4paper,12pt]{report}
\usepackage{xeCJK}
\usepackage{indentfirst}
\usepackage[utf8]{inputenc}
\usepackage{tikz}
\usetikzlibrary{matrix}
\usepackage{fancyhdr}
\usepackage[dutch]{babel}
\usepackage[babel=true]{csquotes}
\usepackage[T1]{fontenc}
\usepackage[explicit]{titlesec}
\usepackage{xcolor}
\usepackage{charter}

\setCJKmainfont{SimSun}
 
\definecolor{mybluei}{RGB}{105,105,105}
\definecolor{myblueii}{RGB}{105,105,105}
\usepackage{rotating, graphicx}
\usepackage{titlesec}
\usepackage{tcolorbox}
\newcommand\ChapterFont{\rmfamily\selectfont\huge}
\newcommand\SectionFont{\bfseries\rmfamily\selectfont\Large}
 
\titleformat{\chapter}[block]
 {\normalfont\ChapterFont\huge\color{myblueii}}%\raisebox{-0.6\height}
 {\tcbset{colframe=mybluei, boxrule=0.8pt, left=0pt, right=0pt, top=0pt, bottom=0pt}\raisebox{-0.48\height}{\rotatebox{90}{\tcbox[boxsep=4pt, colback= white ]{\color{mybluei}\Large\chaptertitlename}}}\hskip 0.25em\mbox{\tcbox[ boxsep=12pt, colback=mybluei, tcbox raise = -35pt]{\color{white}\bfseries\fontsize{70}{70}\selectfont\thechapter}}}
 {0.5em}
 {#1\vskip0.6ex\endgraf\titlerule[1ex]}[]
 
 \titleformat{name=\chapter,numberless}[block]
 {\normalfont\selectfont\huge\color{myblueii}}
 {}
 {0pt}
 {\parbox[b]{70pt}{\mbox{}}%
 \hspace{15pt}%
 \parbox[b]{\dimexpr\textwidth-15pt}{%
 \raggedright\bfseries#1\vskip6pt%
 }%
 }
 \titleformat{\section}
 {\normalfont\small\sffamily\SectionFont\color{myblueii}}
 {\colorbox{mybluei}{%
        \parbox[c][16pt][c]{40pt}{%
            \centering\textcolor{white}{\SectionFont\Large\rmfamily\thesection}%
        }%
    }%
 }
 {1em}
 {#1}
 [\vspace{-0.755\baselineskip}%
 \color{myblueii}\hspace*{\dimexpr40pt+2\fboxsep\relax}%
 \rule{\dimexpr\textwidth-40pt-2\fboxsep\relax}{1pt}%
 ]
  \usepackage{amsmath, amsthm, amssymb, amsfonts, enumerate}%, dlfltxbcodetips
 \renewcommand{\qedsymbol}{$\blacksquare$}

\titleformat{\section}[block]
  {\normalfont\huge\bfseries}
  {\tikz\node[
      font=\huge\bfseries\color{white},
      fill=gray!50,
      rounded corners=20pt,
      minimum height=1.6cm,
      text width=3em,
      align=center,
      inner xsep=0pt] {\parbox{1.5em}{\thechapter\hfill}};%
  }
  {-1em}
  {\tikz\node[
      fill=gray,
      font=\Large\sffamily\color{white},
      minimum height=1.6cm,
      text width=\the\dimexpr\textwidth-2em\relax,
      align=center,inner xsep=0pt] {#1};%
  }
 
\title{OpenStack部署手册}
\date{2016 年 03 月 16 日}
\author{罗燕}
 
\begin{document}
 
\maketitle
 
\renewcommand*\contentsname{Summary}
\tableofcontents

 
\part{First Part of this document}
 
\chapter{First}
 
\section{Introduction}

Lorem ipsum dolor sit amet, consectetur adipiscing elit, sed do 
eiusmod tempor incididunt ut labore et dolore magna aliqua. Ut 
enim ad minim veniam, quis nostrud exercitation ullamco...


Lorem ipsum dolor sit amet, consectetur adipiscing elit, sed do 
eiusmod tempor incididunt ut labore et dolore magna aliqua. Ut 
enim ad minim veniam, quis nostrud exercitation ullamco...

\chapter{ZeroMQ}
 
\section{Summery}

"We use ZeroMQ to assist in aggregating thousands of events occurring every minute across our global
network of telecommunications servers so that we can accurately report and monitor for situations that
require our attention. ZeroMQ made the development of the system not only easier, but faster to develop
and more robust and fault-tolerant than we had originally planned in our original design.
"We’re able to easily add and remove clients from the network without the loss of any message. If we
need to enhance the server portion of our system, we can stop and restart it as well without having to
worry about stopping all of the clients first. The built-in buffering of ZeroMQ makes this all possible."
"I was looking at creating some kind of service bus connecting all kinds of services together. There were
already some products that implemented a broker, but they did not have the functionality I needed. By
accident, I stumbled upon ZeroMQ, which is awesome. It’s very lightweight, lean, simple and easy to
follow because the guide is very complete and reads very well. I’ve actually implemented the Titanic
pattern and the Majordomo broker with some additions (client/worker authentication and workers
sending a catalog explaining what they provide and how they should be addressed).
"The beautiful thing about ZeroMQ is the fact that it is a library and not an application. You can mold it
however you like and it simply puts boring things like queuing, reconnecting, TCP sockets and such to
the background, making sure you can concentrate on what is important to you. I’ve implemented all
kinds of workers/clients and the broker in Ruby, because that is the main language we use for
development, but also some PHP clients to connect to the bus from existing PHP webapps. We use this
service bus for cloud services, connecting all kinds of platform devices to a service bus exposing
functionality for automation.
"ZeroMQ is very easy to understand and if you spend a day with the guide, you’ll have good knowledge
of how it works. I’m a network engineer, not a software developer, but managed to create a very nice
solution for our automation needs! ZeroMQ: Thank you very much!"

\newpage 

\section{Winner}
"We use ZeroMQ to assist in aggregating thousands of events occurring every minute across our global
network of telecommunications servers so that we can accurately report and monitor for situations that
require our attention. ZeroMQ made the development of the system not only easier, but faster to develop
and more robust and fault-tolerant than we had originally planned in our original design.
"We’re able to easily add and remove clients from the network without the loss of any message. If we
need to enhance the server portion of our system, we can stop and restart it as well without having to
worry about stopping all of the clients first. The built-in buffering of ZeroMQ makes this all possible."
"I was looking at creating some kind of service bus connecting all kinds of services together. There were
already some products that implemented a broker, but they did not have the functionality I needed. By
accident, I stumbled upon ZeroMQ, which is awesome. It’s very lightweight, lean, simple and easy to
follow because the guide is very complete and reads very well. I’ve actually implemented the Titanic
pattern and the Majordomo broker with some additions (client/worker authentication and workers
sending a catalog explaining what they provide and how they should be addressed).
"The beautiful thing about ZeroMQ is the fact that it is a library and not an application. You can mold it
however you like and it simply puts boring things like queuing, reconnecting, TCP sockets and such to
the background, making sure you can concentrate on what is important to you. I’ve implemented all
kinds of workers/clients and the broker in Ruby, because that is the main language we use for
development, but also some PHP clients to connect to the bus from existing PHP webapps. We use this
service bus for cloud services, connecting all kinds of platform devices to a service bus exposing
functionality for automation.
"ZeroMQ is very easy to understand and if you spend a day with the guide, you’ll have good knowledge
of how it works. I’m a network engineer, not a software developer, but managed to create a very nice
solution for our automation needs! ZeroMQ: Thank you very much!"
 
\end{document}
