\documentclass[oneside]{book}
\usepackage{xeCJK}
\usepackage{indentfirst}
\usepackage{fancyhdr}
\usepackage{minted}
\usepackage{geometry}
\usepackage{hyperref}
\usepackage{graphicx}
\usepackage[utf8]{inputenc}

\usemintedstyle{native}
\hypersetup{%
    bookmarks=true,
    % 设置文档属性的标题
    pdftitle={VMWare 设计文档},
    % 设置文档属性的作者
    pdfauthor={北京海云捷迅科技有限公司}
}

\setCJKmainfont{SimSun}
\geometry{left=2.5cm,right=2.5cm,top=2.5cm,bottom=2.5cm}

\pagestyle{fancy}
\fancyhf{}
\lhead{VMWare设计文档}
\rhead{\includegraphics[scale=0.02]{cloud.png}}
\renewcommand{\headrulewidth}{0.5pt}
\renewcommand{\footrulewidth}{0.5pt}
\cfoot{\thepage}

% 设置chapter章节页的页眉页脚
\fancypagestyle{plain}{
    \fancyhf{}
    \lhead{VMWare设计文档}
    \rhead{\includegraphics[scale=0.02]{cloud.png}}
	\renewcommand{\headrulewidth}{0.5pt}
	\renewcommand{\footrulewidth}{0.5pt}
    \cfoot{\thepage}
}

\begin{document}

\begin{titlepage}
	\centering{
	{\fontsize{40}{48}\selectfont VMWare 设计文档} }\\
	\vspace{\fill}
	\centering{\Large{北京海云捷迅科技有限公司}}\\
	\vspace{80mm}
\end{titlepage}

\newpage
\frontmatter
{
  \hypersetup{%
      pdfborder = {0 0 0},
      colorlinks,
      citecolor=red,
      filecolor=green,
      linkcolor=myblueii,
      urlcolor=cyan!50!black!90
  }
  \renewcommand*\contentsname{目录}
  \tableofcontents%
  % 不在目录页显示页眉页脚
  \thispagestyle{empty}
}

% 页码从正文开始计算,不从目录开始
\mainmatter

\chapter{快照管理}
\section{创建快照}
\begin{itemize}
\item HTTP方法
\mint{http}|POST|
\item URL
\mint{http}|/v2/{tenant_id}/servers/{server_id}/os-vmware-snapshot|
\item 参数
\begin{minted}{json}
{
    "os-vmware-snapshot": {
        "desc": "zhangbb"                             # 快照的描述
    }
}
\end{minted}
\item 返回值
\begin{minted}{json}
无
\end{minted}
\end{itemize}

\section{查询快照}
\begin{itemize}
\item HTTP方法
\mint{http}|GET|
\item URL
\mint{http}|/v2/{tenant_id}/servers/{server_id}/os-vmware-snapshot|
\item 参数
\begin{minted}{json}
无
\end{minted}
\item 返回值
\begin{minted}{json}
{
    "snapshots": {
        "1c32169a748243ccb182b92715155c2c": {             # 快照ID
            "id": "2",
            "quiesced": "False",
            "state": "poweredOff",                        # 快照状态
            "description": "first",                       # 快照描述
            "createTime": "2016-06-23 18:20:59.679318"    # 快照的创建时间
        },
        "a141027e42e84ebb9d6ef23e7e334080": {
            "quiesced": "False",
            "state": "poweredOff",
            "description": "second",
            "parent": "1c32169a748243ccb182b92715155c2c", # 父快照的id
            "id": "3",
            "createTime": "2016-06-23 18:21:35.108599"
        }
    }
}
\end{minted}
\end{itemize}

\section{删除快照}
\begin{itemize}
\item HTTP方法
\mint{http}|DELETE|
\item URL
\mint{http}|/v2/{tenant_id}/servers/{server_id}/os-vmware-snapshot/{snapshot_id}|
\item 参数
\begin{minted}{json}
无
\end{minted}
\item 返回值
\begin{minted}{json}
无
\end{minted}
\end{itemize}

\section{回滚快照}
\begin{itemize}
\item HTTP方法
\mint{http}|POST|
\item URL
\mint{http}|/v2/{tenant_id}/servers/{server_id}/os-vmware-snapshot/{snapshot_id}/revert|
\item 参数
\begin{minted}{json}
{
    "os-vmware-snapshot": {
        "start_after_revert": true                # 回滚之后,是否打开vm电源
    }
}
\end{minted}
\item 返回值
\begin{minted}{json}
无
\end{minted}
\end{itemize}

\chapter{克隆管理}
\section{克隆VM}
\begin{itemize}
\item HTTP方法
\mint{http}|POST|
\item URL
\mint{http}|/v2/{project_id}/servers/{server_id}/os-server-clone/clone|
\item 参数
\begin{minted}{json}
{
    "os-server-clone": {
        "name": "copy", # 必须参数,克隆出来的虚拟机的名字
        "network_id": "fec51b66-62fa-4176-8891-398a37af3742",
                        # 必须参数,克隆的虚拟机使用的网络id
        "availability_zone": "nova:compute1:domain-1(cluster1)",
                        # 可选参数,与nova boot当中的availability_zone意义一致
        "preferred_esxi": "192.168.1.84",
                        # 可选参数,指定esxi主机
        "preferred_datastore": "datastore1"
                        # 可选参数,指定datastore
    }
}
\end{minted}
\item 返回值
\begin{minted}{json}
无
\end{minted}
\item 特殊说明
\begin{minted}{json}
preferred_esxi和preferred_datastore参数要么同时传递,要么同时都不传递。如果以上
2个参数只传递了一个参数,则clone的结果,是随机选择esxi和datastore进行,不会报错。
\end{minted}
\item 使用限制
\begin{minted}{json}
由于vmware在clone操作的时候,针对虚拟机的存储设备是进行的全拷贝,即如果原始虚拟
机存在3块磁盘,则clone之后的虚拟机也会拥有3块磁盘。这在openstack环境中遇到了麻
烦:虚拟机所有的额外添加的磁盘,都是由cinder提供并管理的。如果在nova层面对多块
盘的虚拟机进行了clone操作,则会出现一个问题:clone出来的虚拟机的额外磁盘,无法
被任何oepnstack项目管理。因此,在openstack环境下,做了部分的限制:不允许对挂载
了volume的虚拟机进行clone操作。
\end{minted}
\end{itemize}

\end{document}
