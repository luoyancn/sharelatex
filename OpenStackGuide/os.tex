\section{基本的安装}

操作系统要求是redhat7/centos7的操作系统
\begin{itemize}
  \item 64位操作系统
  \item 计算节点需要支持虚拟化
  \item 需要 >= 1台以上的服务器或者虚拟机
\end{itemize}

\section{操作系统配置}
\subsection{配置yum源}

\begin{long-block}{bash}
yum install https://repos.fedorapeople.org/repos/openstack/openstack-liberty/rdo-release-liberty-2.noarch.rpm -y

cat >/etc/yum.repos.d/cern.repo<<EOF
[cern-os]
name=cern-os
baseurl=http://linuxsoft.cern.ch/cern/centos/7/os/x86_64
gpgcheck=1
enabled=1
protect=1
priority=5
gpgkey=http://linuxsoft.cern.ch/cern/centos/7/os/x86_64/RPM-GPG-KEY-CentOS-7
[cern-centosplus]
name=cern-centosplus
baseurl=http://linuxsoft.cern.ch/cern/centos/7/centosplus/x86_64
gpgcheck=1
enabled=1
protect=1
priority=5
[cern-cern]
name=cern-cern
baseurl=http://linuxsoft.cern.ch/cern/centos/7/cern/x86_64/
gpgcheck=1
enabled=1
protect=1
priority=5
gpgkey=http://linuxsoft.cern.ch/cern/centos/7/os/x86_64/RPM-GPG-KEY-cern
[cern-extra]
name=cern-extra
baseurl=http://linuxsoft.cern.ch/cern/centos/7/extras/x86_64/
gpgcheck=1
enabled=1
protect=1
priority=5
gpgkey=http://linuxsoft.cern.ch/cern/centos/7/os/x86_64/RPM-GPG-KEY-cern
[cern-update]
name=cern-update
baseurl=http://linuxsoft.cern.ch/cern/centos/7/updates/x86_64/
gpgcheck=1
enabled=1
protect=1
priority=5
gpgkey=http://linuxsoft.cern.ch/cern/centos/7/os/x86_64/RPM-GPG-KEY-cern
[cern-cr]
name=cern-cr
baseurl=http://linuxsoft.cern.ch/cern/centos/7/cr/x86_64/
gpgcheck=1
enabled=1
protect=1
priority=5
gpgkey=http://linuxsoft.cern.ch/cern/centos/7/os/x86_64/RPM-GPG-KEY-CentOS-7
[cern-rt]
name=cern-rt
baseurl=http://linuxsoft.cern.ch/cern/centos/7/rt/x86_64/
gpgcheck=1
enabled=1
protect=1
priority=5
gpgkey=http://linuxsoft.cern.ch/cern/centos/7/os/x86_64/RPM-GPG-KEY-cern
[cern-rhcommon]
name=cern-rhcommon
baseurl=http://linuxsoft.cern.ch/cern/centos/7/rhcommon/x86_64/
gpgcheck=1
enabled=1
protect=1
priority=5
gpgkey=http://linuxsoft.cern.ch/cern/centos/7/os/x86_64/RPM-GPG-KEY-cern
[cern-epel]
name=cern-epel
baseurl=http://linuxsoft.cern.ch/epel/7/x86_64/
gpgcheck=1
enabled=1
protect=1
priority=5
gpgkey=http://linuxsoft.cern.ch/epel/RPM-GPG-KEY-EPEL-7
EOF

yum update -y
yum install openvswitch -y
\end{long-block}

\subsection{关闭selinux}
\begin{code-block}{bash}
sed -i 's/SELINUX=enforcing/SELINUX=disabled/g' /etc/openldap/slapd.conf
\end{code-block}

\subsection{修改主机名}
\begin{code-block}{bash}
echo $hostname > /etc/hostname
\end{code-block}

\subsection{设置OVS开机启动}
\begin{code-block}{bash}
systemctl enable openvswitch
systemctl start openvswitch
\end{code-block}
    
\subsection{修改网络节点网卡配置}
\begin{code-block}{bash}
ovs-vsctl add-br br-ex
cat >/etc/sysconfig/network-scripts/ifcfg-br-ex<<EOF
TYPE=Ethernet
BOOTPROTO=static
DEFROUTE=yes
PEERDNS=no
PEERROUTES=no
IPV4_FAILURE_FATAL=no
IPV6INIT=no
NAME=br-ex
ONBOOT=yes
DEVICE=br-ex
IPADDR=10.7.0.2
NETMASK=255.255.255.0
GATEWAY=10.7.0.254
DEVICETYPE=ovs
TYPE=OVSBridge
EOF
cat >/etc/sysconfig/network-scripts/ifcfg-eth0<<EOF
TYPE=Ethernet
BOOTPROTO=static
DEFROUTE=yes
PEERDNS=no
PEERROUTES=no
IPV4_FAILURE_FATAL=no
IPV6INIT=no
NAME=eth0
ONBOOT=yes
DEVICE=eth0
DEVICETYPE=ovs
OVS_BRIDGE=br-ex
TYPE=OVSPort
EOF
\end{code-block}

\subsection{调整内核参数}
\begin{code-block}{bash}
#网络节点的内核参数
cat >>/etc/sysctl.conf<<EOF
net.ipv4.ip_forward=1
net.ipv4.conf.all.rp_filter=0
net.ipv4.conf.default.rp_filter=0
EOF
#计算节点的内核参数
cat >>/etc/sysctl.conf<<EOF
net.ipv4.conf.all.rp_filter=0
net.ipv4.conf.default.rp_filter=0
net.bridge.bridge-nf-call-iptables=1
net.bridge.bridge-nf-call-ip6tables=1
EOF
\end{code-block}

\subsection{修改操作系统连接数}
\begin{code-block}{bash}
cat >>/etc/security/limits.conf<<EOF
*               soft    core            65536
*               hard    rss             65536
*               soft    nofile          1024000
*               hard    nofile          1024000
*               soft    nproc           1024000
*               hard    nproc           1024000
EOF
\end{code-block}

\subsection{修改操作系统防火墙}
\begin{code-block}{bash}
cat >/etc/sysconfig/iptables<<EOF
*filter
:INPUT ACCEPT [0:0]
EOF
:FORWARD ACCEPT [0:0]
:OUTPUT ACCEPT [0:0]
COMMIT
\end{code-block}

\noindent
完成之后,重启物理机或者服务器
\justifying