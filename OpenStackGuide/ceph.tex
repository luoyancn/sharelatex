\section{Ceph的安装和配置}

\subsection{安装Ceph}

\begin{code-block}{bash}
yum install ceph -y
\end{code-block}

\subsection{修改Ceph配置文件}
Ceph集群的最低要求
    Ceph至少需要一个monitor,3个osd,monitor一定要是奇数个。
    每台ceph osd节点至少需要额外的一块未格式化的磁盘。
    另外,所有的ceph节点的配置文件最好都是完全一致的。

\begin{mdframed}[linecolor=black, topline=true, bottomline=true,
  leftline=false, rightline=false, userdefinedwidth=\textwidth]
\begin{minted}
[fontsize=\footnotesize]
{bash}
cat >/etc/ceph/ceph.conf<<EOF
[global]
fsid = a7f64266-0894-4f1e-a635-d0aeaca0e993
public network = 10.10.2.0/24
cluster network = 10.10.2.0/24
auth cluster required = cephx
auth service required = cephx
auth client required = cephx
filestore xattr use omap = true
osd pool default size = 2
osd pool default min size = 1
osd pool default pg num = 333
osd pool default pgp num = 333
osd crush chooseleaf type = 1
mon osd full ratio = .80
mon osd nearfull ratio = .70

debug lockdep = 0/0
debug context = 0/0
debug crush = 0/0
debug buffer = 0/0
debug timer = 0/0
debug journaler = 0/0
debug osd = 0/0
debug optracker = 0/0
debug objclass = 0/0
debug filestore = 0/0
debug journal = 0/0
debug ms = 0/0
debug monc = 0/0
debug tp = 0/0
debug auth = 0/0
debug finisher = 0/0
debug heartbeatmap = 0/0
debug perfcounter = 0/0
debug asok = 0/0
debug throttle = 0/0

[mon]
mon initial members = controller,compute1,compute2
mon host = 10.10.2.101:6789, 10.10.2.102:6789, 10.10.2.103:6789

[mon.controller]
host = controller
mon addr = 10.10.2.101:6789
mon data = /var/lib/ceph/mon/ceph-controller

[mon.compute1]
host = compute1
mon addr = 10.10.2.102:6789
mon data = /var/lib/ceph/mon/ceph-compute1

[mon.compute2]
host = compute2
mon addr = 10.10.2.103:6789
mon data = /var/lib/ceph/mon/ceph-compute2

[osd]
osd journal size = 1024
osd data = /var/lib/ceph/osd/$cluster-$id
osd journal = /var/lib/ceph/osd/$cluster-$id/journal

[osd.0]
osd host = controller
public addr = 10.10.2.101
cluster addr = 10.10.2.101

[osd.1]
osd host = compute1
public addr = 10.10.2.102
cluster addr = 10.10.2.102

[osd.2]
osd host = compute2
public addr = 10.10.2.103
cluster addr = 10.10.2.103
EOF
\end{minted}
\end{mdframed}

\subsection{创建ceph monitor key和admin key}

\begin{code-block}{bash}
ceph-authtool --create-keyring /tmp/ceph.mon.keyring --gen-key \
    -n mon. --cap mon 'allow *'

ceph-authtool --create-keyring /etc/ceph/ceph.client.admin.keyring \
    --gen-key -n client.admin --set-uid=0 --cap mon 'allow *' \
    --cap osd 'allow *' --cap mds 'allow'

ceph-authtool /tmp/ceph.mon.keyring \
    --import-keyring /etc/ceph/ceph.client.admin.keyring

monmaptool --create --add controller 10.10.2.101  \
                    --add compute1 10.10.2.102 \
                    --add compute2 10.10.2.103  \
                    --fsid a7f64266-0894-4f1e-a635-d0aeaca0e993 /tmp/monmap
\end{code-block}

\subsection{创建ceph monitor}
\begin{code-block}{bash}
mkdir -p /var/lib/ceph/mon/ceph-controller
ceph-mon --mkfs -i controller --monmap /tmp/monmap \
    --keyring /tmp/ceph.mon.keyring
service ceph start mon

scp /etc/ceph/* root@compute1:/etc/ceph
scp /tmp/monmap root@compute1:/tmp
scp -r /tmp/ceph.mon root@compute1:/tmp/
ssh compute1
mkdir -p /var/lib/ceph/mon/ceph-compute1
ceph-mon --mkfs -i compute1 --monmap /tmp/monmap \
    --keyring /tmp/ceph.mon.keyring
service ceph start mon

scp /etc/ceph/* root@compute2:/etc/ceph
scp /tmp/monmap root@compute2:/tmp
scp -r /tmp/ceph.mon root@compute2:/tmp/
ssh compute2
mkdir -p /var/lib/ceph/mon/ceph-compute2
ceph-mon --mkfs -i compute2 --monmap /tmp/monmap \
    --keyring /tmp/ceph.mon.keyring
service ceph start mon
\end{code-block}

\subsection{创建ceph osd}
从monitor节点拷贝如下目录到各osd节点
/etc/ceph
/var/lib/ceph/bootstrap-osd

\begin{code-block}{bash}
mkdir -p /var/lib/ceph/osd/ceph-0
ceph-disk prepare --cluster ceph \
    --cluster-uuid a7f64266-0894-4f1e-a635-d0aeaca0e993 \
    --fs-type xfs  /dev/sdb
ceph-disk activate /dev/sdb1
    
ssh compute1
mkdir -p /var/lib/ceph/osd/ceph-1
ceph-disk prepare --cluster ceph \
    --cluster-uuid a7f64266-0894-4f1e-a635-d0aeaca0e993 \
    --fs-type xfs  /dev/sdb
ceph-disk activate /dev/sdb1
    
ssh compute2
mkdir -p /var/lib/ceph/osd/ceph-2
ceph-disk prepare --cluster ceph \
    --cluster-uuid a7f64266-0894-4f1e-a635-d0aeaca0e993 \
    --fs-type xfs  /dev/sdb
ceph-disk activate /dev/sdb1
\end{code-block}

\subsection{创建ceph poolh和相关用户}

\begin{code-block}{bash}
ceph osd pool create volumes 3 3 replicated
ceph auth get-or-create client.awcloud mon 'allow r' \
    osd 'allow class-read object_prefix rbd_children,allow rwx pool=volumes'
ceph auth get-or-create client.awcloud | tee /etc/ceph/ceph.client.awcloud.keyring
\end{code-block}

\subsection{导入ceph到libvirt}

\begin{code-block}{bash}
export secret_id=9d257295-27ea-4810-4106-f0e9d0ead209
cat > secret.xml <<EOF
    <secret ephemeral='no' private='no'>
      <uuid>$secret_id</uuid>
      <usage type='ceph'>
        <name>client.awcloud</name>
      </usage>
    </secret>
EOF
virsh secret-define --file secret.xml
ceph auth get-key client.awcloud | tee client.awcloud.key
virsh secret-set-value --secret $secret_id --base64 $(cat client.awcloud.key)
\end{code-block}