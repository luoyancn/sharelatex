\section{Libvirt配置}
\subsection{创建CA证书}

\begin{code-block}{bash}
mkdir -p /etc/pki/libvirt/private;
mkdir -p /opt/libvirttls;
cd /opt/libvirttls
certtool --generate-privkey > cakey.pem
cat >ca.info<<EOF
cn = Awcloud
ca
cert_signing_key
expiration_days = 3650
EOF
certtool --generate-self-signed --load-privkey cakey.pem \
         --template ca.info --outfile cacert.pem
\end{code-block}

将生成的cakey.pem和cacert.pem复制到所有node节点的/etc/pki/CA下

\subsection{创建服务器证书}

\begin{code-block}{bash}
mkdir -p /etc/pki/libvirt/private;
mkdir -p /opt/libvirttls;
cd /opt/libvirttls
certtool --generate-privkey > serverkey.pem
export host=`hostname`
cat> server.info<<EOF
organization = Awcloud
cn = $host
tls_www_server
encryption_key
signing_key
expiration_days = 3650
EOF
certtool --generate-certificate --load-privkey serverkey.pem \
         --load-ca-certificate /etc/pki/CA/cacert.pem \
         --load-ca-privkey /etc/pki/CA/cakey.pem \
         --template server.info \
         --outfile servercert.pem
mv servercert.pem /etc/pki/libvirt;
mv serverkey.pem /etc/pki/libvirt/private;
\end{code-block}

\subsection{制作客户端证书}

\begin{code-block}{bash}
mkdir -p /etc/pki/libvirt/private;
mkdir -p /opt/libvirttls;
cd /opt/libvirttls
certtool --generate-privkey > clientkey.pem
export host=`hostname`
cat> client.info<<EOF
country = CN
state = HB
locality = Beijing
organization = Awcloud
cn = $host
tls_www_client
encryption_key
signing_key
expiration_days = 3650
EOF
certtool --generate-certificate --load-privkey clientkey.pem \
         --load-ca-certificate /etc/pki/CA/cacert.pem \
         --load-ca-privkey /etc/pki/CA/cakey.pem \
         --template client.info \
         --outfile clientcert.pem
mv clientcert.pem /etc/pki/libvirt;
mv clientkey.pem /etc/pki/libvirt/private;
\end{code-block}

\subsection{修改libvirt配置参数}

\begin{code-block}{bash}
sed -i 's/#LIBVIRTD_ARGS="--listen"/LIBVIRTD_ARGS="--listen"/g' \
    /etc/sysconfig/libvirtd
sed -i 's/#listen_tls = 0/listen_tls = 1/g' /etc/libvirt/libvirtd.conf
sed -i 's/#tls_port = "16514"/tls_port = "16514"/g' /etc/libvirt/libvirtd.conf
sed -i 's/#tls_no_verify_certificate = 1/tls_no_verify_certificate = 1/g' \
    /etc/libvirt/libvirtd.conf
sed -i 's/#ca_file = "\/etc\/pki\/CA\/cacert.pem"/\
ca_file = "\/etc\/pki\/CA\/cacert.pem"/g' \
    /etc/libvirt/libvirtd.conf
sed -i 's/#cert_file = "\/etc\/pki\/libvirt\/servercert.pem"/\
cert_file = "\/etc\/pki\/libvirt\/servercert.pem"/g' \
    /etc/libvirt/libvirtd.conf
sed -i 's/#key_file = "\/etc\/pki\/libvirt\/private\/serverkey.pem"/\
key_file = "\/etc\/pki\/libvirt\/private\/serverkey.pem"/g' \
    /etc/libvirt/libvirtd.conf
\end{code-block}

\subsection{重启并验证libvirt}
\begin{code-block}{bash}
systemctl restart libvirtd
netstat -tnal | grep 16514
virsh -c qemu+tls://<hostname>/qemu
\end{code-block}
如果16514端口监听正常,则说明libvirt配置成功;如果virsh连接正常,则说明libvirt的互信配置正常